\def\shapeCard{(0,0) rectangle (\cardwidth,\cardheight)}
\def\shapeInnerBorder{(0.25,0.25) rectangle (\cardwidth-0.25,\cardheight-0.25)}

\tikzset{
	cardcorners/.style={
		rounded corners=0.2cm
	}
}

\newcommand{\cardborder}{
	\draw[dark_grey,cardcorners] \shapeCard;
}

\newcommand{\cardMiddle}{
	\draw[dark_grey] (0.15*\cardwidth,0.5*\cardheight) -- (0.85*\cardwidth,0.5*\cardheight);
}

\newcommand{\coloredBorder}[1]{
	\clip[cardcorners] \shapeInnerBorder;
	\draw[line width=0.2cm,#1,cardcorners] \shapeInnerBorder;
}
\newcommand{\circleCorners}[1]{
	\clip \shapeInnerBorder;
	\draw[line width=0.1cm,#1,fill=white] (0.25,0.25) circle (0.5cm);
	\draw[line width=0.1cm,#1,fill=white] (0.25,\cardheight-0.25) circle (0.5cm);
	\draw[line width=0.1cm,#1,fill=white] (\cardwidth-0.25,0.25) circle (0.5cm);
	\draw[line width=0.1cm,#1,fill=white] (\cardwidth-0.25,\cardheight-0.25) circle (0.5cm);
}
\newcommand{\squareCorners}[1]{
	\clip \shapeInnerBorder;
	\draw[line width=0.1cm,#1,fill=white] (0,0) rectangle (0.75,0.75);
	\draw[line width=0.1cm,#1,fill=white] (0,\cardheight) rectangle (0.75,\cardheight-0.75);
	\draw[line width=0.1cm,#1,fill=white] (\cardwidth,0) rectangle (\cardwidth-0.75,0.75);
	\draw[line width=0.1cm,#1,fill=white] (\cardwidth,\cardheight-0) rectangle (\cardwidth-0.75,\cardheight-0.75);
}
\newcommand{\triangleCorners}[1]{
	\clip \shapeInnerBorder;
	\draw[line width=0.1cm,#1,fill=white] (0,0) -- (0,1) -- (1,0) -- cycle;
	\draw[line width=0.1cm,#1,fill=white] (\cardwidth,0) -- (\cardwidth,1) -- (\cardwidth-1,0) -- cycle;
	\draw[line width=0.1cm,#1,fill=white] (\cardwidth,\cardheight) -- (\cardwidth,\cardheight-1) -- (\cardwidth-1,\cardheight) -- cycle;
	\draw[line width=0.1cm,#1,fill=white] (0,\cardheight) -- (0,\cardheight-1) -- (1,\cardheight) -- cycle;
}

\newcommand{\suitSquare}[1]{
	\fill [#1]    (0.5*\cardwidth-0.5, \cardheight-0.5)
		rectangle (0.5*\cardwidth+0.5, \cardheight-1.5);
	\fill [#1]    (0.5*\cardwidth-0.5, 0.5)
		rectangle (0.5*\cardwidth+0.5, 1.5);
}

\newcommand{\suitTriangle}[1]{
	\fill [#1] (0.5*\cardwidth-0.6, \cardheight-1.5)
			-- (0.5*\cardwidth+0.6, \cardheight-1.5)
			-- (0.5*\cardwidth, \cardheight-0.5)
			-- cycle;
	\fill [#1] (0.5*\cardwidth-0.6, 1.5)
			-- (0.5*\cardwidth+0.6, 1.5)
			-- (0.5*\cardwidth, 0.5)
			-- cycle;
}

\newcommand{\suitCircle}[1]{
	\fill [#1] (0.5*\cardwidth, \cardheight-1) circle (0.5cm);
	\fill [#1] (0.5*\cardwidth, 1) circle (0.5cm);
}

\newcommand{\cardvalue}[2]{
	\node [text width=1cm] at (1, 1.2) {
		\begin{center}
			\color{#2}\rotatebox{180}{\Huge \textbf #1}
		\end{center}
	};
	\node [text width=1cm] at (\cardwidth-1, 1.2) {
		\begin{center}
			\color{#2}\rotatebox{180}{\Huge \textbf #1}
		\end{center}
	};
	\node [text width=1cm] at (1, \cardheight-1) {
		\begin{center}
			\color{#2}{\Huge \textbf #1}
		\end{center}
	};
	\node [text width=1cm] at (\cardwidth-1, \cardheight-1) {
		\begin{center}
			\color{#2}{\Huge \textbf #1}
		\end{center}
	};
}

\newcommand{\effect}[1] {
	\node [text width=4cm] at (0.5*\cardwidth, 0.5*\cardheight+1.25) {
		\parbox[c]{4cm}{\centering \large #1}
	};
	\node [text width=4cm] at (0.5*\cardwidth, 0.5*\cardheight-1.25) {
		\rotatebox{180}{\parbox[c]{4cm}{\centering \large #1}}
	};
}

